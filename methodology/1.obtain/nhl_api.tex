%! Author = Stepan Oskin
%! Date = 2019-07-19

% Preamble
\documentclass[11pt]{article}

% Packages
\usepackage{hyperref}
\hypersetup{
colorlinks=true,
linkcolor=blue,
filecolor=magenta,
urlcolor=cyan,
}

\urlstyle{same}

% Document
\begin{document}

    \title{NHL dataset \\
    Obtaining data from the NHL API}

    \author{Stepan Oskin}

    \maketitle

    \begin{abstract}
        This document describes the process of obtaining data from NHL Records and Stats APIs.
    \end{abstract}

    \section{NHL API} \label{sec:nhl_api}

    NHL provides two main APIs: Stats API and Records API\@.
    Description of some of their endpoints is provided in the following sections \\
    (taken from \textit{Philip Bulsink} \url{https://gitlab.com/dword4/nhlapi}).


    \section{NHL Stats API endpoints} \label{sec:nhl_stats}

    Only some endpoints are presented here, full description can be found here: \url{https://gitlab.com/dword4/nhlapi/blob/master/stats-api.md}.

    \begin{itemize}
        \item \textbf{Teams} \\
        \textbf{GET} \url{https://statsapi.web.nhl.com/api/v1/teams} \\
        Returns a list of data about all teams including their id, venue details, division, conference and franchise information.

        \textbf{GET} \url{https://statsapi.web.nhl.com/api/v1/teams/ID/roster} \\
        Returns entire roster for a team including id value, name, jersey number and position details.

        \item \textbf{People} \\
        \textbf{GET} \url{https://statsapi.web.nhl.com/api/v1/people/ID} \\
        Gets details for a player, must specify the id value in order to return data.

        \textbf{GET} \url{https://statsapi.web.nhl.com/api/v1/people/ID/stats} \\
        Complex endpoint with lots of append options to change what kind of stats you wish to obtain

        \textit{Modifiers} \\
        \url{?stats=statsSingleSeason&season=19801981} \\
        Obtains single season statistics for a player

        \url{?stats=homeAndAway&season=20162017} \\
        Provides a split between home and away games.

        \url{?stats=winLoss&season=20162017} \\
        Very similar to the previous modifier except it provides the W/L/OT split instead of Home and Away

        \url{?stats=byMonth&season=20162017} \\
        Monthly split of stats

        \url{?stats=byDayOfWeek&season=20162017} \\
        Split done by day of the week

        \url{?stats=vsDivision&season=20162017} \\
        Division stats split

        \url{?stats=vsConference&season=20162017} \\
        Conference stats split

        \url{?stats=vsTeam&season=20162017} \\
        Conference stats split

        \url{?stats=gameLog&season=20162017} \\
        Provides a game log showing stats for each game of a season

        \url{?stats=regularSeasonStatRankings&season=20162017} \\
        Returns where someone stands vs the rest of the league for a specific \texttt{regularSeasonStatRankings}

        \url{?stats=goalsByGameSituation&season=20162017} \\
        Shows number on when goals for a player happened like how many in the shootout, how many in each period, etc.

        \url{?stats=onPaceRegularSeason&season=20172018} \\
        This only works with the current in-progress season and shows projected totals based on current \texttt{onPaceRegularSeason}

        \item \textbf{Draft} \\
        \textbf{GET} \url{https://statsapi.web.nhl.com/api/v1/draft} \\
        Get round-by-round data for current year's NHL Entry Draft.

        \textbf{GET} \url{https://statsapi.web.nhl.com/api/v1/draft/YEAR} \\
        Takes a YYYY format year and returns draft data

        \item \textbf{Prospects} \\
        \textbf{GET} \url{https://statsapi.web.nhl.com/api/v1/draft/prospects} \\
        Get all NHL Entry Draft prospects.

        \textbf{GET} \url{https://statsapi.web.nhl.com/api/v1/draft/prospects/ID} \\
        Get an NHL Entry Draft prospect.

    \end{itemize}


    \section{NHL Records API endpoints} \label{sec:nhl_records}

    Only some endpoints are presented here, full description can be found here: \url{https://gitlab.com/dword4/nhlapi/blob/master/records-api.md}.

    All queries are prefixed with https://records.nhl.com/site/api and are GET requests unless otherwise noted.

    \begin{itemize}
        \item \textbf{Filtering} \\

        This is slightly different than the normal NHL API, see the following example: \\
        \url{https://records.nhl.com/site/api/draft?cayenneExp=draftYear=2017\%20and\%20draftedByTeamId=15}

        The \texttt{\%20} value translates to a space, this needs to be taken into account as removing the spaces will break the query, so anything after \texttt{cayenneExp} can have spaces when separating two or more conditions.

        Often you can filter by information returned in an unfiltered query so using the draft example you can append \texttt{roundNumber=4} onto the \texttt{cayenneExp} to only look at 4th round selections.

        \item \textbf{Draft}
        \textbf{GET} \url{https://records.nhl.com/site/api/draft} \\
        Returns A LOT of draft data, looks to be every pick ever.

        \textit{Filtering} \\
        \url{?cayenneExp=draftYear=2017} \\
        This filters by a single year.

        \url{draftedByTeamId=ID} \\
        Drill down to a specific teams drafting.

    \end{itemize}

\end{document}