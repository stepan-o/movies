%! Author = Stepan Oskin
%! Date = 2019-07-19

% Preamble
\documentclass[11pt]{article}

% Packages
\usepackage{hyperref}
\hypersetup{
colorlinks=true,
linkcolor=blue,
filecolor=magenta,
urlcolor=cyan,
}

\urlstyle{same}

% Document
\begin{document}

    \title{Movies dataset \\
    OSEMN methodology, Step 1: \\
    Obtaining movies data from various sources}

    \author{Stepan Oskin}

    \maketitle

    \begin{abstract}
        This document describes the process of obtaining movies data from various sources.
    \end{abstract}

    \section{Data on Entertainment} \label{sec:ent_data}

    La Fabbrica della Realta, an Entertainment Marketing Consulting agency, provides a good overview of available data sources for entertainment industry:

    \url{https://www.lafabbricadellarealta.com/open-data-entertainment/}

    \vspace{5mm}

    Of the data that allows the movie and TV industries to function, very little is open.
    From the list of sources that feature more than 1'000 items provided by La Fabbrica della Realta, some sources have a proper open data license, while most of them don't.

    Below is the description of some of the sources listed by La Fabbrica della Realta which have been chosen for the purposes of this analysis.

    \section{IMDb Datasets} \label{sec:imdb}

    IMDb datasets provide a trove of troves of entertainment data that are refreshed daily and available for \textbf{personal and non-commercial use}.
    You are allowed to hold local copies of this data, and it is subject to IMBb terms and conditions.

    \newpage

    \subsection{IMDb Licence} \label{subsec:imbd_licence}

    Compliance can be verified by referring to:
    \begin{itemize}

        \item Non-Commercial Licensing

        \url{https://help.imdb.com/article/imdb/general-information/can-i-use-imdb-data-in-my-software/G5JTRESSHJBBHTGX?pf_rd_m=A2FGELUUNOQJNL&pf_rd_p=3aefe545-f8d3-4562-976a-e5eb47d1bb18&pf_rd_r=MBQYD9CR3MEXPZJ2PSQS&pf_rd_s=center-1&pf_rd_t=60601&pf_rd_i=interfaces&ref_=fea_mn_lk1}

        \item copyright/licence

        \url{http://www.imdb.com/Copyright?pf_rd_m=A2FGELUUNOQJNL&pf_rd_p=3aefe545-f8d3-4562-976a-e5eb47d1bb18&pf_rd_r=MBQYD9CR3MEXPZJ2PSQS&pf_rd_s=center-1&pf_rd_t=60601&pf_rd_i=interfaces&ref_=fea_mn_lk2}

    \end{itemize}

    \subsection{IMDb Data Location} \label{subsec:imdb_data_loc}

    The dataset files can be accessed and downloaded from:

    \url{https://datasets.imdbws.com/}

    The data is refreshed daily.

    \subsection{IMDb Dataset Details} \label{subsec:imdb_data_desc}

    Each dataset is contained in a gzipped, tab-separated-values (TSV) formatted file in the UTF-8 character set.
    The first line in each file contains headers that describe what is in each column.
    A \texttt{\textbackslash N} is used to denote that a particular field is missing or null for that title/name.

    \vspace{5mm}
    The available datasets are as follows:
    \vspace{5mm}

    \textbf{title.akas.tsv.gz} - Contains the following information for titles:

    \begin{itemize}
        \item \texttt{titleId} (string)\textemdash a tconst, an alphanumeric unique identifier of the title
        \item \texttt{ordering} (integer)\textemdash a number to uniquely identify rows for a given \texttt{titleId}
        \item \texttt{title} (string)\textemdash the localized title
        \item \texttt{region} (string)\textemdash the region for this version of the title
        \item \texttt{language} (string)\textemdash the language of the title
        \item \texttt{types} (array)\textemdash Enumerated set of attributes for this alternative title.
        One or more of the following: "alternative", "dvd", "festival", "tv", "video", "working", "original", "imdbDisplay".
        New values may be added in the future without warning
        \item \texttt{attributes} (array)\textemdash Additional terms to describe this alternative title, not enumerated
        \item \texttt{isOriginalTitle} (boolean)\textemdash 0: not original title;
        1: original title
    \end{itemize}

    \textbf{title.basics.tsv.gz}\textemdash Contains the following information for titles:

    \begin{itemize}
        \item \texttt{tconst} (string)\textemdash alphanumeric unique identifier of the title
        \item \texttt{titleType} (string)\textemdash the type/format of the title (\textit{e.g.,} movie, short, tvseries, tvepisode, video, \textit{etc.})
        \item \texttt{primaryTitle} (string)\textemdash the more popular title / the title used by the filmmakers on promotional materials at the point of release
        \item \texttt{originalTitle} (string)\textemdash original title, in the original language
        \item \texttt{isAdult} (boolean)\textemdash 0: non-adult title;
        1: adult title
        \item \texttt{startYear} (YYYY)\textemdash represents the release year of a title.
        In the case of TV Series, it is the series start year
        \item \texttt{endYear} (YYYY)\textemdash TV Series end year.
        \texttt{\textbackslash N} for all other title types
        \item \texttt{runtimeMinutes}\textemdash primary runtime of the title, in minutes
        \item \texttt{genres} (string array)\textemdash includes up to three genres associated with the title
    \end{itemize}

    \textbf{title.crew.tsv.gz}\textemdash Contains the director and writer information for all the titles in IMDb. Fields include:

    \begin{itemize}
        \item \texttt{tconst} (string)\textemdash alphanumeric unique identifier of the title
        \item \texttt{directors} (array of nconsts)\textemdash director(s) of the given title
        \item \texttt{writers} (array of nconsts)\textemdash writer(s) of the given title
    \end{itemize}

    \newpage

    \textbf{title.episode.tsv.gz}\textemdash Contains the tv episode information.
    Fields include:

    \begin{itemize}
        \item \texttt{tconst} (string)\textemdash alphanumeric identifier of episode
        \item \texttt{parentTconst} (string)\textemdash alphanumeric identifier of the parent TV Series
        \item \texttt{seasonNumber} (integer)\textemdash season number the episode belongs to
        \item \texttt{episodeNumber} (integer)\textemdash episode number of the \texttt{tconst} in the TV series
    \end{itemize}

    \textbf{title.principals.tsv.gz}\textemdash Contains the principal cast/crew for titles

    \begin{itemize}
        \item \texttt{tconst} (string)\textemdash alphanumeric unique identifier of the title
        \item \texttt{ordering} (integer)\textemdash a number to uniquely identify rows for a given \texttt{titleId}
        \item \texttt{nconst} (string)\textemdash alphanumeric unique identifier of the name/person
        \item \texttt{category} (string)\textemdash the category of job that person was in
        \item \texttt{job} (string)\textemdash the specific job title if applicable, else \texttt{\textbackslash N}
        \item \texttt{characters} (string)\textemdash the name of the character played if applicable, else \texttt{\textbackslash N}
    \end{itemize}

    \textbf{title.ratings.tsv.gz}\textemdash Contains the IMDb rating and votes information for titles

    \begin{itemize}
        \item \texttt{tconst} (string)\textemdash alphanumeric unique identifier of the title
        \item \texttt{averageRating}\textemdash weighted average of all the individual user ratings
        \item \texttt{numVotes}\textemdash number of votes the title has received
    \end{itemize}

    \textbf{name.basics.tsv.gz}\textemdash Contains the following information for names:

    \begin{itemize}
        \item \texttt{nconst} (string)\textemdash alphanumeric unique identifier of the name/person
        \item \texttt{primaryName} (string)\textemdash name by which the person is most often credited
        \item \texttt{birthYear}\textemdash in YYYY format
        \item \texttt{deathYear}\textemdash in YYYY format if applicable, else \texttt{\textbackslash N}
        \item \texttt{primaryProfession} (array of strings)\textemdash the top-3 professions of the person
        \item \texttt{knownForTitles} (array of tconsts)\textemdash titles the person is known for
    \end{itemize}

    \section{Other data sources} \label{sec:other_data}

    Below are some other movie data sources that can be added at a later stage.

    \begin{itemize}

        \item \textbf{OMDb API} \\
        The Open Movie Database \\
        The OMDb API is a RESTful web service to obtain movie information, all content and images on the site are contributed and maintained by our users. \\
        \url{https://www.omdbapi.com}

        \item \textbf{MovieLens} \\
        GroupLens Research has collected and made available rating data sets from the MovieLens web site (\url{http://movielens.org}).
        The data sets were collected over various periods of time, depending on the size of the set.
        Before using these data sets, please review their README files for the usage licenses and other details. \\
        \url{https://grouplens.org/datasets/movielens/}

        \item \textbf{Movie Review Data} \\
        This page is a distribution site for movie-review data for use in sentiment-analysis experiments.
        Available are collections of movie-review documents labeled with respect to their overall sentiment polarity (positive or negative) or subjective rating (e.g., "two and a half stars") and sentences labeled with respect to their subjectivity status (subjective or objective) or polarity. \\
        \url{http://www.cs.cornell.edu/people/pabo/movie-review-data/}

        \item \textbf{UCI ML repository Movie Data Set} \\
        This data set contains a list of over 10000 films including many older, odd, and cult films.
        There is information on actors, casts, directors, producers, studios, etc. \\
        \url{https://archive.ics.uci.edu/ml/datasets/Movie}

    \end{itemize}

\end{document}